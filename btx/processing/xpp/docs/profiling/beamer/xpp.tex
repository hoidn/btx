\documentclass{beamer}
\usepackage{listings}
\usepackage{xcolor}
\usepackage{fontawesome}
\usepackage{tikz}  % Add this
\usepackage{pgf-pie}  % Add this for pie charts

\definecolor{codegreen}{rgb}{0,0.6,0}
\definecolor{codegray}{rgb}{0.5,0.5,0.5}
\definecolor{backcolour}{rgb}{0.95,0.95,0.92}

\definecolor{codegreen}{rgb}{0,0.6,0}
\definecolor{codegray}{rgb}{0.5,0.5,0.5}
\definecolor{backcolour}{rgb}{0.95,0.95,0.92}

\lstdefinestyle{mystyle}{
    backgroundcolor=\color{backcolour},   
    commentstyle=\color{codegreen},
    basicstyle=\ttfamily\small,
    breakatwhitespace=false,
    breaklines=true,
    captionpos=b,
    keepspaces=true,
    showspaces=false,
    showstringspaces=false,
    showtabs=false,
    tabsize=2
}

\lstset{style=mystyle}


\title{Differences Between LCLS Dataset Analysis Workflows}

\begin{document}

%\begin{frame}
%\titlepage
%\end{frame}

\begin{frame}{Pipeline Runtime Analysis}
\begin{columns}
\column{0.6\textwidth}
\begin{itemize}
\item Total runtime: 26.91s
\item Main stages:
    \begin{itemize}
    \item \textcolor{red}{LoadData: 11.40s (42.3\%)}
        \begin{itemize}
        \item Energy threshold: 11.38s
        \item Delay binning: 0.02s
        \end{itemize}
    \item \textcolor{blue}{MakeHistogram: 11.27s (41.9\%)}
        \begin{itemize}
        \item Histogram calculation: 11.27s
        \end{itemize}
    \item \textcolor{green}{Signal Analysis: 2.72s (10.1\%)}
        \begin{itemize}
        \item EMD + P-values + Masks: 1.20s
        \item PumpProbe: 1.52s
        \end{itemize}
    \end{itemize}
\end{itemize}

\column{0.4\textwidth}
%\begin{tikzpicture}
%\pie[radius=2]{
%    42.3/LoadData,
%    41.9/MakeHistogram,
%    10.1/Analysis,
%    5.7/Other
%}
%\end{tikzpicture}
\end{columns}

\vspace{0.3cm}
\begin{block}{Key Bottlenecks}
\begin{itemize}
\item Energy threshold computation in LoadData
\item Histogram calculation with large frame count (37,301)
\end{itemize}
\end{block}
\end{frame}

\begin{frame}{Data loading differences}
\begin{itemize}
\item Two distinct analysis workflows for different datasets
\item Key differences in:
    \begin{itemize}
    \item HDF5 path structures
    \item Event classification methods
    \item Detector mask handling
    \item Filter parameters
    \item Data processing approaches
    \end{itemize}
\end{itemize}
\end{frame}

\begin{frame}{HDF5 Path Structure - Delay Encoding}
\textbf{Dataset A: xppx1003221}
\begin{itemize}
\item Uses \texttt{enc/lasDelay} combined with timetool position
\item Additional timing corrections applied
\end{itemize}

\textbf{Dataset B: xppl1030522}
\begin{itemize}
\item Uses \texttt{enc/lasDelay2} directly
\item Different delay calculation methodology
\end{itemize}

\textbf{Impact:} Different delay calculations affect time binning and resolution
\end{frame}

\begin{frame}[fragile]{Event Classification}
\textbf{Dataset A:}
\begin{lstlisting}[language=Python]
# HDF5 attributes:
/evr/code_90             # uint8 array
/evr/code_91             # uint8 array
/evr/code_40             # uint8 array

# Processing:
laser_on = evr.code_90 == 1
laser_off = evr.code_91 == 1
\end{lstlisting}
\end{frame}

\begin{frame}[fragile]{Event Classification (cont.)}
\textbf{Dataset B:}
\begin{lstlisting}[language=Python]
# HDF5 attributes:
/lightStatus/laser       # bool array
/lightStatus/xray        # bool array
/lightStatus/valid       # bool array

# Processing:
laser_on = lightStatus/laser == True
xray_on = lightStatus/xray == True
\end{lstlisting}
\end{frame}

\begin{frame}[fragile]{Detector Mask Handling}
\textbf{Dataset A:}
\begin{lstlisting}[language=Python]
# HDF5 attributes:
/UserDataCfg/jungfrau1M/ROI_0__ROI_0_ROI
/UserDataCfg/jungfrau1M/mask  

# Processing:
idx_tile = ROI_0__ROI_0_ROI[0,0]
mask = jungfrau1M.mask[idx_tile][roi_slice_y, roi_slice_x]
\end{lstlisting}
\end{frame}

\begin{frame}[fragile]{Detector Mask Handling (cont.)}
\textbf{Dataset B:}
\begin{lstlisting}[language=Python]
# HDF5 attributes:
/UserDataCfg/jungfrau1M/ROI_0__ROI_0_mask

# Processing:
roi0_mask = ROI_0__ROI_0_mask[0]
\end{lstlisting}

\textbf{Impact:} Different approaches to background subtraction and signal isolation
\end{frame}

\begin{frame}{Filter Parameters}
\textbf{IPM Position Filters:}
\begin{itemize}
\item \textbf{Dataset A:}
    \begin{itemize}
    \item X: [-0.2, 0.2]
    \item Y: [-0.5, 0.5]
    \end{itemize}
\item \textbf{Dataset B:}
    \begin{itemize}
    \item X: [-0.45, 0.45]
    \item Y: [-1.6, 0.0]
    \end{itemize}
\end{itemize}
\end{frame}

\begin{frame}{TimeTool Integration}
\begin{columns}
\column{0.5\textwidth}
\textbf{Dataset A}
\begin{itemize}
\item Optional usage
\item Configurable threshold
\item Laser-on events only
\end{itemize}

\column{0.5\textwidth}
\textbf{Dataset B}
\begin{itemize}
\item Always enabled
\item Fixed threshold
\item All events
\end{itemize}
\end{columns}
\end{frame}

\end{document}
